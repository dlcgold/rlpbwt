% \documentclass[handout]{beamer}
\documentclass{beamer}
\mode<presentation>
{
  \usetheme{Warsaw}
  % \setbeamercovered{transparent}
  \useoutertheme{infolines}

}

\usepackage[italian]{babel}
\usepackage[utf8]{inputenc}
\usepackage{wrapfig}
\usepackage[T1]{fontenc}
\usepackage{float}
\usepackage{multicol}
\usepackage{blindtext}
\usepackage{mwe}

\definecolor{nord1}{RGB}{46, 52, 64} 
\definecolor{nord2}{RGB}{76, 105, 141} 
\definecolor{nord3}{RGB}{94, 129, 172}
\definecolor{nord4}{RGB}{129, 161, 193} 
\definecolor{nord5}{RGB}{136, 192, 208}  
\definecolor{nord6}{RGB}{163, 190, 140}
\definecolor{nord7}{RGB}{191, 97, 106}
\definecolor{nordred}{RGB}{191, 97, 106}
\definecolor{nordgreen}{RGB}{135, 157, 116}

\setbeamercolor{palette primary}{bg=nord2,fg=white}
\setbeamercolor{palette secondary}{bg=nord3,fg=white}
\setbeamercolor{palette tertiary}{bg=nord4,fg=white}

\setbeamercolor{block title}{bg=nord2,fg=white}

\setbeamercolor{itemize item}{fg=nord3}
\setbeamercolor{itemize subitem}{fg=nord4}
\setbeamercolor{itemize subsubitem}{fg=nord5}

\setbeamertemplate{itemize item}[square]
\setbeamertemplate{itemize subitem}[circle]
\setbeamertemplate{itemize subsubitem}[triangle]

\usecolortheme[named=nord2]{structure}

\title[] {RLPBWT}

% \subtitle
% {Presentation Subtitle} % (optional)

\author[] {Davide Cozzi}


\institute[] {Dipartimento di Informatica, Sistemistica e Comunicazione
  (DISCo)\\
  Università degli Studi di Milano Bicocca}

\date[] {}

\subject{Presentation}
\pgfdeclareimage[height=0.5cm]{university-logo}{img/logo_unimib.pdf}
\logo{\pgfuseimage{university-logo}}

\AtBeginSection[]
{
  \begin{frame}<beamer>{Outline}
    \tableofcontents[currentsection, currentsubsection]
  \end{frame}
}


% If you wish to uncover everything in a step-wise fashion, uncomment
% the following command: 

% \beamerdefaultoverlayspecification{<+->}


\begin{document}

\begin{frame}
  \titlepage
\end{frame}

\begin{frame}{Outline}
  \setcounter{tocdepth}{1}
  \tableofcontents
\end{frame}

\section{RLPBWT}
\begin{frame}{Some definitions}
  \begin{block}{The permutation, panel $M$, $n\times m$}
    In \textit{RLPBWT} we have a permutation $\pi_j$, $\forall\,\, 1\leq j\leq
    m$ that stably sorts the bits of the $j$-th column of the PBWT.\\
    This permutation can be stored in space
    proportional to the number of runs in the $j$-th column of the PBWT
  \end{block}
  \pause
  \begin{block}{}
    The positions in the columns of the PBWT of the bits in the $i$-th row of
    $M$ are:
    \pause
    \[i, \pi_1(i), \pi_2(\pi_1(i)),\ldots, \pi_{m-1}(\cdots(
      \pi_2(\pi_1(i)))\cdots)\] 
  \end{block}
  \begin{block}{}
    Extracting the bits of the $i$-th row of $M$ reduces to iteratively applying
    the $\pi_{m-1}$ permutations, corresponding to iteratively apply LF in a
    standard BWT 
  \end{block}
\end{frame}
\begin{frame}{The compressed data structure}
  \begin{block}{The tables}
    \begin{itemize}
      \item a set of $m$ tables in which the $m$-th table stores only the
      positions of the run-heads in the $m$-th column and a bool to check the
      first symbol: 0 or 1
      \item the $i$-th row of the $j$-th table stores a quadruple
    \end{itemize}    
  \end{block}
  \pause
  \begin{block}{The quadruple}
    \begin{enumerate}
      \item the position $p$ of the $i$-th run-head in the $j$-th column of the
      PBWT 
      \item the permutation $\pi_j(p)$
      \item the index of the run containing bit $\pi_j(p)$ in the $(j + 1)$-st
      column of the PBWT
      \item the threshold, that's the index of the minimum \textit{LCP value}
      (current column minus divergence array value) in the run
    \end{enumerate}
  \end{block}
\end{frame}
\begin{frame}{Row extraction}
  \begin{block}{First step}
    We start by finding the row of the first table that starts with the position
    $p$ of the head of the run containing bit $i$ in first column of the PBWT,
    computing: 
    \pause
    \[\pi_1(i)=\pi_1(p)+i-p\]
    \pause
    looking up the row for the run containing bit $\pi_1(p)$ in the the second
    table and scanning down the table until we find the row for the run
    containing bit $\pi_1(i)$ 
  \end{block}
  \begin{block}{Next step}
    We continue repeating this procedure for each column
  \end{block}
  
\end{frame}
\section{Example}
\begin{frame}{The Panel}
  \begin{block}{Panel}
    \begin{table}[H]
      \centering
      \tiny
      \begin{tabular}{c|c|c|c|c|c|c|c|c|c|c|c|c|c|c|c|c|c|c|c}
        \hline
        0 & 1 & 2 & 3 & 4 & 5 & 6 & 7 & 8 & 9 & 10 & 11 & 12 & 13 & 14 & 15 & 16
        & 17 & 18 & 19\\
        \hline
        \hline
        
        1 & 1 & 0 & 1 & 0 & 0 & 1 & 1 & 1 & 1 & 1 & 0 & 0 & 1 & 0 & 0 & 1 & 0
             & 0 & 1\\
        0 & 1 & 0 & 0 & 0 & 0 & 1 & 1 & 1 & 1 & 1 & 0 & 0 & 1 & 1 & 1 & 0 & 0
             & 1 & 0\\
        0 & 0 & 0 & 1 & 0 & 0 & 0 & 0 & 0 & 1 & 1 & 1 & 1 & 0 & 0 & 0 & 1 & 0
             & 1 & 0\\
        1 & 0 & 0 & 1 & 1 & 0 & 1 & 0 & 1 & 0 & 0 & 0 & 1 & 1 & 1 & 0 & 0 & 0
             & 1 & 0\\
        0 & 1 & 1 & 0 & 1 & 1 & 1 & 1 & 1 & 0 & 0 & 1 & 0 & 0 & 1 & 1 & 1 & 1
             & 0 & 0\\
        1 & 1 & 0 & 0 & 1 & 0 & 1 & 0 & 1 & 0 & 1 & 0 & 1 & 0 & 0 & 0 & 1 & 1
             & 1 & 1\\
        0 & 0 & 0 & 1 & 0 & 1 & 1 & 1 & 1 & 1 & 1 & 1 & 0 & 0 & 1 & 0 & 0 & 0
             & 1 & 1\\
        \hline
      \end{tabular}
    \end{table}
  \end{block}
  \begin{block}{PBWT Matrix}
    \begin{table}[H]
      \centering
      \tiny
      \begin{tabular}{c|c|c|c|c|c|c|c|c|c|c|c|c|c|c|c|c|c|c|c}
        \hline
        0 & 1 & 2 & 3 & 4 & 5 & 6 & 7 & 8 & 9 & 10 & 11 & 12 & 13 & 14 & 15 & 16
        & 17 & 18 & 19\\
        \hline
        \hline
        1 & 1 & 0 & 1 & 0 & 0 & 1 & 0 & 0 & 1 & 1 & 0 & 1 & 1 & 1 & 0 & 1 & 0
             & 1 & 1 \\
        0 & 0 & 0 & 1 & 1 & 0 & 0 & 1 & 1 & 0 & 0 & 1 & 1 & 1 & 1 & 0 & 1 & 0
             & 1 & 0 \\
        0 & 1 & 0 & 1 & 1 & 1 & 1 & 1 & 1 & 0 & 0 & 0 & 0 & 0 & 0 & 0 & 1 & 0
             & 1 & 1 \\
        1 & 0 & 0 & 0 & 0 & 0 & 1 & 0 & 1 & 1 & 1 & 1 & 0 & 0 & 0 & 1 & 0 & 1
             & 1 & 0 \\
        0 & 1 & 1 & 1 & 0 & 0 & 1 & 0 & 1 & 1 & 1 & 0 & 0 & 1 & 1 & 0 & 0 & 0
             & 0 & 0 \\
        1 & 0 & 0 & 0 & 1 & 1 & 1 & 1 & 1 & 1 & 1 & 0 & 1 & 0 & 0 & 1 & 1 & 0
             & 1 & 0 \\
        0 & 1 & 0 & 0 & 0 & 0 & 1 & 1 & 1 & 0 & 1 & 1 & 0 & 0 & 1 & 0 & 0 & 1
             & 0 & 1 \\
        \hline
      \end{tabular}
    \end{table}
  \end{block}
\end{frame}
\begin{frame}{Prefix and Divergence Arrays}
  \begin{block}{Prefix Arrays}
    \begin{table}[H]
      \tiny
      \begin{tabular}{c|c|c|c|c|c|c|c|c|c|c|c|c|c|c|c|c|c|c|c}
        \hline
        0 & 1 & 2 & 3 & 4 & 5 & 6 & 7 & 8 & 9 & 10 & 11 & 12 & 13 & 14 & 15 & 16
        & 17 & 18 & 19\\
        \hline
        \hline
        0 & 1 & 2 & 2 & 1 & 1 & 1 & 2 & 2 & 2 & 5 & 3 & 3 & 1 & 4 & 5 & 5 & 6
             & 6 & 0\\  
        1 & 2 & 6 & 6 & 5 & 2 & 2 & 1 & 5 & 5 & 3 & 4 & 5 & 0 & 6 & 2 & 2 & 3
             & 3 & 4\\  
        2 & 4 & 3 & 3 & 4 & 6 & 0 & 0 & 3 & 3 & 4 & 5 & 1 & 4 & 5 & 0 & 0 & 1
             & 1 & 6\\  
        3 & 6 & 1 & 1 & 2 & 0 & 5 & 5 & 1 & 1 & 2 & 2 & 0 & 6 & 2 & 4 & 6 & 5
             & 2 & 3\\  
        4 & 0 & 4 & 0 & 6 & 5 & 3 & 3 & 0 & 0 & 1 & 1 & 4 & 3 & 1 & 6 & 3 & 2
             & 0 & 1\\  
        5 & 3 & 0 & 5 & 3 & 4 & 6 & 6 & 6 & 6 & 0 & 0 & 2 & 5 & 0 & 1 & 4 & 0
             & 5 & 2\\
        6 & 5 & 5 & 4 & 0 & 3 & 4 & 4 & 4 & 4 & 6 & 6 & 6 & 2 & 3 & 3 & 1 & 4
             & 4 & 5\\
        \hline
      \end{tabular}
    \end{table}
  \end{block}
  \begin{block}{LCP Arrays: current \textit{k} minus the original Durbin's
      divergence arrays}  
    \begin{table}[H]
      \tiny
      \begin{tabular}{c|c|c|c|c|c|c|c|c|c|c|c|c|c|c|c|c|c|c|c}
        \hline
        0 & 1 & 2 & 3 & 4 & 5 & 6 & 7 & 8 & 9 & 10 & 11 & 12 & 13 & 14 & 15 & 16
        & 17 & 18 & 19\\
        \hline
        \hline
        0 & 0 & 0 & 0 & 0 & 0 & 0 & 0 & 0 & 0 & 0 & 0 & 0 & 0 & 0 & 0 & 0 & 0
             & 0 & 0 \\
        0 & 1 & 2 & 3 & 3 & 1 & 2 & 0 & 1 & 0 & 6 & 3 & 1 & 9 & 3 & 3 & 4 & 3
             & 4 & 1 \\
        0 & 1 & 1 & 2 & 1 & 5 & 4 & 3 & 4 & 5 & 2 & 0 & 2 & 1 & 1 & 1 & 2 & 1
             & 2 & 0 \\
        0 & 1 & 0 & 1 & 0 & 3 & 1 & 2 & 0 & 1 & 0 & 1 & 8 & 2 & 2 & 0 & 1 & 0
             & 1 & 5 \\
        0 & 0 & 2 & 2 & 4 & 0 & 2 & 3 & 4 & 5 & 1 & 2 & 0 & 0 & 0 & 4 & 2 & 5
             & 4 & 3 \\
        0 & 1 & 1 & 3 & 3 & 2 & 0 & 1 & 2 & 3 & 6 & 7 & 1 & 2 & 10 & 1 & 0 & 3
             & 0 & 2 \\
        0 & 1 & 2 & 0 & 2 & 1 & 1 & 2 & 3 & 4 & 4 & 5 & 3 & 1 & 1 & 2 & 2 & 1
             & 2 & 1\\
      \end{tabular}
    \end{table}
  \end{block}
\end{frame}

\begin{frame}{Run-Length PBWT I}
  \begin{block}{Some tables: \textit{p, perm, next perm, threshold}}
    {\footnotesize{\[
          \begin{matrix}
            0 & 4 & 4 & 0\\
            1 & 0 & 0 & 1\\
            3 & 5 & 5 & 3\\
            \color{nordgreen} 4 &\color{nordgreen} 2 &\color{nordgreen} 2
            &\color{nordgreen} 4\\ 
            5 & 6 & 6 & 5\\
            6 & 3 & 3 & 6
          \end{matrix}
          \Longrightarrow
          \begin{matrix}
            0 & 3 & 0 & 0\\
            1 & 0 & 0 & 1\\
            \color{nordgreen} 2 &  \color{nordgreen} 4 &  \color{nordgreen} 1 &
            \color{nordgreen} 2\\
            3 & 1 & 0 & 3\\
            4 & 5 & 2 & 4\\
            5 & 2 & 0 & 5\\
            6 & 6 & 2 & 6
          \end{matrix}
          \Longrightarrow
          \begin{matrix}
            0 & 0 & 0 & 0\\
            \color{nordgreen} 4 & \color{nordgreen} 6 & \color{nordgreen} 3 &
            \color{nordgreen} 4\\
            5 & 4 &  2 & 5
          \end{matrix}
          \Longrightarrow
          \begin{matrix}
            0 & 3 & 2 & 0\\
            3 & 0 & 0 & 3\\
            4 & 6 & 4 & 4\\
            \color{nordgreen} 5 & \color{nordgreen} 1 & \color{nordgreen} 1 &
            \color{nordgreen} 6
          \end{matrix}
          \Longrightarrow
          \begin{matrix}
            0 & 0 & 0 & 0\\
            \color{nordgreen} 1 &  \color{nordgreen} 4 &  \color{nordgreen} 2 &
            \color{nordgreen} 2\\
            3 & 1 & 0 & 3\\
            5 & 6 & 4 & 5\\
            6 & 3 & 2 & 6
          \end{matrix}\]}}
  \end{block}
  \begin{block}{}
    {\footnotesize{\[
          \begin{matrix}
            0 & 0 & 0 & 0\\
            2 & 5 & 2 & 2\\
            \color{nordred} 3 & \color{nordred} 2 & \color{nordred} 2 &
            \color{nordred} 4\\
            \color{nordgreen} 5 &  \color{nordgreen} 6 &   \color{nordgreen}2 &
            \color{nordgreen} 5\\
            6 & 4 & 2 & 6
          \end{matrix}
          \Longrightarrow
          \begin{matrix}
            0 & 1 & 1 & 0\\
            1 & 0 & 0 & 1\\
            \color{nordgreen} 2 & \color{nordgreen} 2 & \color{nordgreen} 1 &
            \color{nordgreen} 5
          \end{matrix}
          \Longrightarrow
          \begin{matrix}
            0 & 0 & 0 & 0\\
            \color{nordred} 1 &  \color{nordred} 3 &  \color{nordred} 1 &
            \color{nordred} 1\\ 
            3 & 1 & 1 & 3\\
            \color{nordgreen} 5 & \color{nordgreen} 5 & \color{nordgreen} 1 &
            \color{nordgreen} 5
          \end{matrix}
          \Longrightarrow
          \begin{matrix}
            0 & 0 & 0 & 0\\
            \color{nordgreen} 1 & \color{nordgreen} 1 & \color{nordgreen} 1 &
            \color{nordgreen} 3
          \end{matrix}
          \Longrightarrow
          \begin{matrix}
            0 & 3 & 2 & 0\\
            \color{nordred} 1 &  \color{nordred} 0 &  \color{nordred} 0 &
            \color{nordred} 1\\ 
            3 & 4 & 2 & 3\\
            \color{nordgreen} 6 & \color{nordgreen} 2 & \color{nordgreen} 1 &
            \color{nordgreen} 6           
          \end{matrix}\]}}
  \end{block}
\end{frame}
\begin{frame}{Run-Length PBWT II}
  \begin{block}{}
    {\footnotesize{\[
          \begin{matrix}
            0 & 2 & 2 & 0\\
            \color{nordgreen} 1 & \color{nordgreen} 0 & \color{nordgreen} 0 &
            \color{nordgreen} 2\\
            3 & 3 & 3 & 3
          \end{matrix}
          \Longrightarrow
          \begin{matrix}
            \color{nordred} 0 & \color{nordred} 0 & \color{nordred} 0 &
            \color{nordred} 0\\
            \color{nordgreen} 1 & \color{nordgreen} 4 & \color{nordgreen} 1 &
            \color{nordgreen} 1\\
            2 & 1 & 0 & 2\\
            3 & 5 & 2 & 3\\
            4 & 2 & 1 & 4\\
            6 & 6 & 3 & 6
          \end{matrix}
          \Longrightarrow
          \begin{matrix}
            0 & 4 & 2 & 0\\
            \color{nordgreen} 2 & \color{nordgreen} 0 & \color{nordgreen} 0 &
            \color{nordgreen} 4\\
            5 & 6 & 3 & 5\\
            6 & 3 & 1 & 6
          \end{matrix}
          \Longrightarrow
          \begin{matrix}
            \color{nordred} 0 & \color{nordred} 4 & \color{nordred} 2 &
            \color{nordred} 0\\
            \color{nordgreen} 2 & \color{nordgreen} 0 & \color{nordgreen} 0 &
            \color{nordgreen}2\\ 
            4 & 6 & 4 & 4\\
            5 & 2 & 1 & 6
          \end{matrix}
          \Longrightarrow
          \begin{matrix}
            \color{nordgreen} 0 & \color{nordgreen} 3 & \color{nordgreen} 1 &
            \color{nordgreen} 0\\ 
            2 & 0 & 0 & 2\\
            4 & 5 & 3 & 4\\
            5 & 2 & 0 & 5\\
            6 & 6 & 4 & 6
          \end{matrix}\]}}
  \end{block}
  \begin{block}{}
    {\footnotesize{\[
          \begin{matrix}
            0 & 0 & 0 & 0\\
            \color{nordgreen} 3 & \color{nordgreen} 5 & \color{nordgreen} 2 &
            \color{nordgreen} 3\\
            4 & 3 & 1 & 4\\
            5 & 6 & 3 & 5\\
            6 & 4 & 1 & 6
          \end{matrix}
          \Longrightarrow
          \begin{matrix}
            0 & 3 & 1 & 0\\
            3 & 0 & 0 & 3\\
            \color{nordgreen}5 & \color{nordgreen} 6 & \color{nordgreen} 3 &
            \color{nordgreen} 5\\ 
            6 & 2 & 0 & 6
          \end{matrix}
          \Longrightarrow
          \begin{matrix}
            0 & 0 & 0 & 0\\
            3 & 5 & 2 & 3\\
            4 & 3 & 0 & 5\\
            \color{nordgreen} 6 & \color{nordgreen} 6 & \color{nordgreen} 3 &
            \color{nordgreen} 6
          \end{matrix}
          \Longrightarrow
          \begin{matrix}  
            0 & 2 & 2 & 0\\
            4 & 0 & 0 & 4\\
            5 & 6 & 4 & 5\\
            \color{nordgreen} 6 & \color{nordgreen} 1 & \color{nordgreen} 1 &
            \color{nordgreen} 6
          \end{matrix}
          \Longrightarrow
          \begin{matrix}  
            0 & 4 & 0 & 0\\
            \color{nordgreen} 1 & \color{nordgreen} 0 & \color{nordgreen} 0 &
            \color{nordgreen} 1\\
            2 & 5 & 0 & 2\\
            3 & 1 & 0 & 5\\
            6 & 6 & 0 & 6
          \end{matrix}\]}}
  \end{block}
\end{frame}
% \begin{frame}[allowframebreaks]{References} 
%   \nocite{*}
%   \bibliographystyle{unsrt}
%   \bibliography{ref}
% \end{frame}

\end{document}


