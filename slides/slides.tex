%\documentclass[handout]{beamer}
\documentclass{beamer}
\mode<presentation>
{
  \usetheme{Warsaw}
  % \setbeamercovered{transparent}
  \useoutertheme{infolines}

}

\usepackage[italian]{babel}
\usepackage[utf8]{inputenc}
\usepackage{wrapfig}
\usepackage[T1]{fontenc}
\usepackage{float}
\usepackage{multicol}
\usepackage{blindtext}
\usepackage{mwe}

\definecolor{nord1}{RGB}{46, 52, 64} 
\definecolor{nord2}{RGB}{76, 105, 141} 
\definecolor{nord3}{RGB}{94, 129, 172}
\definecolor{nord4}{RGB}{129, 161, 193} 
\definecolor{nord5}{RGB}{136, 192, 208}  
\definecolor{nord6}{RGB}{163, 190, 140}
\definecolor{nord7}{RGB}{191, 97, 106}

\setbeamercolor{palette primary}{bg=nord2,fg=white}
\setbeamercolor{palette secondary}{bg=nord3,fg=white}
\setbeamercolor{palette tertiary}{bg=nord4,fg=white}

\setbeamercolor{block title}{bg=nord2,fg=white}

\setbeamercolor{itemize item}{fg=nord3}
\setbeamercolor{itemize subitem}{fg=nord4}
\setbeamercolor{itemize subsubitem}{fg=nord5}

\setbeamertemplate{itemize item}[square]
\setbeamertemplate{itemize subitem}[circle]
\setbeamertemplate{itemize subsubitem}[triangle]

\usecolortheme[named=nord2]{structure}

\title[] {RLPBWT}

% \subtitle
% {Presentation Subtitle} % (optional)

\author[] {Davide Cozzi}


\institute[] {Dipartimento di Informatica, Sistemistica e Comunicazione
  (DISCo)\\
  Università degli Studi di Milano Bicocca}

\date[] {}

\subject{Presentation}
\pgfdeclareimage[height=0.5cm]{university-logo}{img/logo_unimib.pdf}
\logo{\pgfuseimage{university-logo}}

\AtBeginSection[]
{
  \begin{frame}<beamer>{Outline}
    \tableofcontents[currentsection, currentsubsection]
  \end{frame}
}


% If you wish to uncover everything in a step-wise fashion, uncomment
% the following command: 

%\beamerdefaultoverlayspecification{<+->}


\begin{document}

\begin{frame}
  \titlepage
\end{frame}

\begin{frame}{Outline}
  \setcounter{tocdepth}{1}
  \tableofcontents
\end{frame}

\section{RLPBWT}
\begin{frame}{Some definitions}

\end{frame}
\section{Example}
\begin{frame}{The Panel}
  \begin{block}{Panel}
    \begin{table}[H]
      \centering
      \tiny
      \begin{tabular}{c|c|c|c|c|c|c|c|c|c|c|c|c|c|c|c|c|c|c|c}
        \hline
        0 & 1 & 2 & 3 & 4 & 5 & 6 & 7 & 8 & 9 & 10 & 11 & 12 & 13 & 14 & 15 & 16
        & 17 & 18 & 19\\
        \hline
        \hline
        
        1 & 1 & 0 & 1 & 0 & 0 & 1 & 1 & 1 & 1 & 1 & 0 & 0 & 1 & 0 & 0 & 1 & 0
             & 0 & 1\\
        0 & 1 & 0 & 0 & 0 & 0 & 1 & 1 & 1 & 1 & 1 & 0 & 0 & 1 & 1 & 1 & 0 & 0
             & 1 & 0\\
        0 & 0 & 0 & 1 & 0 & 0 & 0 & 0 & 0 & 1 & 1 & 1 & 1 & 0 & 0 & 0 & 1 & 0
             & 1 & 0\\
        1 & 0 & 0 & 1 & 1 & 0 & 1 & 0 & 1 & 0 & 0 & 0 & 1 & 1 & 1 & 0 & 0 & 0
             & 1 & 0\\
        0 & 1 & 1 & 0 & 1 & 1 & 1 & 1 & 1 & 0 & 0 & 1 & 0 & 0 & 1 & 1 & 1 & 1
             & 0 & 0\\
        1 & 1 & 0 & 0 & 1 & 0 & 1 & 0 & 1 & 0 & 1 & 0 & 1 & 0 & 0 & 0 & 1 & 1
             & 1 & 1\\
        0 & 0 & 0 & 1 & 0 & 1 & 1 & 1 & 1 & 1 & 1 & 1 & 0 & 0 & 1 & 0 & 0 & 0
             & 1 & 1\\
        \hline
      \end{tabular}
    \end{table}
  \end{block}
  \begin{block}{PBWT Matrix}
    \begin{table}[H]
      \centering
      \tiny
      \begin{tabular}{c|c|c|c|c|c|c|c|c|c|c|c|c|c|c|c|c|c|c|c}
        \hline
        0 & 1 & 2 & 3 & 4 & 5 & 6 & 7 & 8 & 9 & 10 & 11 & 12 & 13 & 14 & 15 & 16
        & 17 & 18 & 19\\
        \hline
        \hline
        1 & 1 & 0 & 1 & 0 & 0 & 1 & 0 & 0 & 1 & 1 & 0 & 1 & 1 & 1 & 0 & 1 & 0
             & 1 & 1 \\
        0 & 0 & 0 & 1 & 1 & 0 & 0 & 1 & 1 & 0 & 0 & 1 & 1 & 1 & 1 & 0 & 1 & 0
             & 1 & 0 \\
        0 & 1 & 0 & 1 & 1 & 1 & 1 & 1 & 1 & 0 & 0 & 0 & 0 & 0 & 0 & 0 & 1 & 0
             & 1 & 1 \\
        1 & 0 & 0 & 0 & 0 & 0 & 1 & 0 & 1 & 1 & 1 & 1 & 0 & 0 & 0 & 1 & 0 & 1
             & 1 & 0 \\
        0 & 1 & 1 & 1 & 0 & 0 & 1 & 0 & 1 & 1 & 1 & 0 & 0 & 1 & 1 & 0 & 0 & 0
             & 0 & 0 \\
        1 & 0 & 0 & 0 & 1 & 1 & 1 & 1 & 1 & 1 & 1 & 0 & 1 & 0 & 0 & 1 & 1 & 0
             & 1 & 0 \\
        0 & 1 & 0 & 0 & 0 & 0 & 1 & 1 & 1 & 0 & 1 & 1 & 0 & 0 & 1 & 0 & 0 & 1
             & 0 & 1 \\
        \hline
      \end{tabular}
    \end{table}
  \end{block}
\end{frame}
\begin{frame}{Prefix and Divergence Arrays}
  \begin{block}{Prefix Arrays}
    \begin{table}[H]
      \tiny
      \begin{tabular}{c|c|c|c|c|c|c|c|c|c|c|c|c|c|c|c|c|c|c|c}
        \hline
        0 & 1 & 2 & 3 & 4 & 5 & 6 & 7 & 8 & 9 & 10 & 11 & 12 & 13 & 14 & 15 & 16
        & 17 & 18 & 19\\
        \hline
        \hline
        0 & 1 & 2 & 2 & 1 & 1 & 1 & 2 & 2 & 2 & 5 & 3 & 3 & 1 & 4 & 5 & 5 & 6
             & 6 & 0\\  
        1 & 2 & 6 & 6 & 5 & 2 & 2 & 1 & 5 & 5 & 3 & 4 & 5 & 0 & 6 & 2 & 2 & 3
             & 3 & 4\\  
        2 & 4 & 3 & 3 & 4 & 6 & 0 & 0 & 3 & 3 & 4 & 5 & 1 & 4 & 5 & 0 & 0 & 1
             & 1 & 6\\  
        3 & 6 & 1 & 1 & 2 & 0 & 5 & 5 & 1 & 1 & 2 & 2 & 0 & 6 & 2 & 4 & 6 & 5
             & 2 & 3\\  
        4 & 0 & 4 & 0 & 6 & 5 & 3 & 3 & 0 & 0 & 1 & 1 & 4 & 3 & 1 & 6 & 3 & 2
             & 0 & 1\\  
        5 & 3 & 0 & 5 & 3 & 4 & 6 & 6 & 6 & 6 & 0 & 0 & 2 & 5 & 0 & 1 & 4 & 0
             & 5 & 2\\
        6 & 5 & 5 & 4 & 0 & 3 & 4 & 4 & 4 & 4 & 6 & 6 & 6 & 2 & 3 & 3 & 1 & 4
             & 4 & 5\\
        \hline
      \end{tabular}
    \end{table}
  \end{block}
  \begin{block}{LCP Arrays: current \textit{k} minus the original Durbin's
      divergence arrays}  
    \begin{table}[H]
      \tiny
      \begin{tabular}{c|c|c|c|c|c|c|c|c|c|c|c|c|c|c|c|c|c|c|c}
        \hline
        0 & 1 & 2 & 3 & 4 & 5 & 6 & 7 & 8 & 9 & 10 & 11 & 12 & 13 & 14 & 15 & 16
        & 17 & 18 & 19\\
        \hline
        \hline
        0 & 0 & 0 & 0 & 0 & 0 & 0 & 0 & 0 & 0 & 0 & 0 & 0 & 0 & 0 & 0 & 0 & 0
             & 0 & 0 \\
        0 & 1 & 2 & 3 & 3 & 1 & 2 & 0 & 1 & 0 & 6 & 3 & 1 & 9 & 3 & 3 & 4 & 3
             & 4 & 1 \\
        0 & 1 & 1 & 2 & 1 & 5 & 4 & 3 & 4 & 5 & 2 & 0 & 2 & 1 & 1 & 1 & 2 & 1
             & 2 & 0 \\
        0 & 1 & 0 & 1 & 0 & 3 & 1 & 2 & 0 & 1 & 0 & 1 & 8 & 2 & 2 & 0 & 1 & 0
             & 1 & 5 \\
        0 & 0 & 2 & 2 & 4 & 0 & 2 & 3 & 4 & 5 & 1 & 2 & 0 & 0 & 0 & 4 & 2 & 5
             & 4 & 3 \\
        0 & 1 & 1 & 3 & 3 & 2 & 0 & 1 & 2 & 3 & 6 & 7 & 1 & 2 & 10 & 1 & 0 & 3
             & 0 & 2 \\
        0 & 1 & 2 & 0 & 2 & 1 & 1 & 2 & 3 & 4 & 4 & 5 & 3 & 1 & 1 & 2 & 2 & 1
             & 2 & 1\\
      \end{tabular}
    \end{table}
  \end{block}
\end{frame}

\begin{frame}{Run-Length PBWT}
  \begin{block}{Some tables: \textit{p, perm, next perm, threshold}}
      {\footnotesize{\[
      \begin{matrix}
        p & pe & np & th\\
        \hline
        0 & 4 & 4 & 0\\
        1 & 0 & 0 & 1\\
        3 & 5 & 5 & 3\\
        4 & 2 & 2 & 4\\
        5 & 6 & 6 & 5\\
        6 & 3 & 3 & 6
      \end{matrix}\Longrightarrow \begin{matrix}
        p & pe & np & th\\
        \hline
        0 & 3 & 0 & 0\\
        1 & 0 & 0 & 1\\
        2 & 4 & 1 & 2\\
        3 & 1 & 0 & 3\\
        4 & 5 & 2 & 4\\
        5 & 2 & 0 & 5\\
        6 & 6 & 2 & 6
      \end{matrix}\Longrightarrow \begin{matrix}
        p & pe & np & th\\
        \hline
        0 & 0 & 0 & 0\\
        4 & 6 & 3 & 4\\
        5 & 4 & 2 & 5
      \end{matrix}\Longrightarrow \begin{matrix}
        p & pe & np & th\\
        \hline
        0 & 3 & 2 & 0\\
        3 & 0 & 0 & 3\\
        4 & 6 & 4 & 4\\
        5 & 1 & 1 & 6
      \end{matrix}\Rightarrow\ldots\]}}
\end{block}
\end{frame}

% \begin{frame}[allowframebreaks]{References} 
%     \nocite{*}
%     \bibliographystyle{unsrt}
%     \bibliography{ref}
% \end{frame}

\end{document}


